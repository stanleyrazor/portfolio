% Options for packages loaded elsewhere
\PassOptionsToPackage{unicode}{hyperref}
\PassOptionsToPackage{hyphens}{url}
\PassOptionsToPackage{dvipsnames,svgnames,x11names}{xcolor}
%
\documentclass[
  letterpaper,
  DIV=11,
  numbers=noendperiod,
  oneside]{scrartcl}

\usepackage{amsmath,amssymb}
\usepackage{iftex}
\ifPDFTeX
  \usepackage[T1]{fontenc}
  \usepackage[utf8]{inputenc}
  \usepackage{textcomp} % provide euro and other symbols
\else % if luatex or xetex
  \usepackage{unicode-math}
  \defaultfontfeatures{Scale=MatchLowercase}
  \defaultfontfeatures[\rmfamily]{Ligatures=TeX,Scale=1}
\fi
\usepackage{lmodern}
\ifPDFTeX\else  
    % xetex/luatex font selection
\fi
% Use upquote if available, for straight quotes in verbatim environments
\IfFileExists{upquote.sty}{\usepackage{upquote}}{}
\IfFileExists{microtype.sty}{% use microtype if available
  \usepackage[]{microtype}
  \UseMicrotypeSet[protrusion]{basicmath} % disable protrusion for tt fonts
}{}
\makeatletter
\@ifundefined{KOMAClassName}{% if non-KOMA class
  \IfFileExists{parskip.sty}{%
    \usepackage{parskip}
  }{% else
    \setlength{\parindent}{0pt}
    \setlength{\parskip}{6pt plus 2pt minus 1pt}}
}{% if KOMA class
  \KOMAoptions{parskip=half}}
\makeatother
\usepackage{xcolor}
\usepackage[left=1in,marginparwidth=2.0666666666667in,textwidth=4.1333333333333in,marginparsep=0.3in]{geometry}
\setlength{\emergencystretch}{3em} % prevent overfull lines
\setcounter{secnumdepth}{-\maxdimen} % remove section numbering
% Make \paragraph and \subparagraph free-standing
\ifx\paragraph\undefined\else
  \let\oldparagraph\paragraph
  \renewcommand{\paragraph}[1]{\oldparagraph{#1}\mbox{}}
\fi
\ifx\subparagraph\undefined\else
  \let\oldsubparagraph\subparagraph
  \renewcommand{\subparagraph}[1]{\oldsubparagraph{#1}\mbox{}}
\fi


\providecommand{\tightlist}{%
  \setlength{\itemsep}{0pt}\setlength{\parskip}{0pt}}\usepackage{longtable,booktabs,array}
\usepackage{calc} % for calculating minipage widths
% Correct order of tables after \paragraph or \subparagraph
\usepackage{etoolbox}
\makeatletter
\patchcmd\longtable{\par}{\if@noskipsec\mbox{}\fi\par}{}{}
\makeatother
% Allow footnotes in longtable head/foot
\IfFileExists{footnotehyper.sty}{\usepackage{footnotehyper}}{\usepackage{footnote}}
\makesavenoteenv{longtable}
\usepackage{graphicx}
\makeatletter
\def\maxwidth{\ifdim\Gin@nat@width>\linewidth\linewidth\else\Gin@nat@width\fi}
\def\maxheight{\ifdim\Gin@nat@height>\textheight\textheight\else\Gin@nat@height\fi}
\makeatother
% Scale images if necessary, so that they will not overflow the page
% margins by default, and it is still possible to overwrite the defaults
% using explicit options in \includegraphics[width, height, ...]{}
\setkeys{Gin}{width=\maxwidth,height=\maxheight,keepaspectratio}
% Set default figure placement to htbp
\makeatletter
\def\fps@figure{htbp}
\makeatother

\KOMAoption{captions}{tableheading}
\makeatletter
\makeatother
\makeatletter
\makeatother
\makeatletter
\@ifpackageloaded{caption}{}{\usepackage{caption}}
\AtBeginDocument{%
\ifdefined\contentsname
  \renewcommand*\contentsname{Table of contents}
\else
  \newcommand\contentsname{Table of contents}
\fi
\ifdefined\listfigurename
  \renewcommand*\listfigurename{List of Figures}
\else
  \newcommand\listfigurename{List of Figures}
\fi
\ifdefined\listtablename
  \renewcommand*\listtablename{List of Tables}
\else
  \newcommand\listtablename{List of Tables}
\fi
\ifdefined\figurename
  \renewcommand*\figurename{Figure}
\else
  \newcommand\figurename{Figure}
\fi
\ifdefined\tablename
  \renewcommand*\tablename{Table}
\else
  \newcommand\tablename{Table}
\fi
}
\@ifpackageloaded{float}{}{\usepackage{float}}
\floatstyle{ruled}
\@ifundefined{c@chapter}{\newfloat{codelisting}{h}{lop}}{\newfloat{codelisting}{h}{lop}[chapter]}
\floatname{codelisting}{Listing}
\newcommand*\listoflistings{\listof{codelisting}{List of Listings}}
\makeatother
\makeatletter
\@ifpackageloaded{caption}{}{\usepackage{caption}}
\@ifpackageloaded{subcaption}{}{\usepackage{subcaption}}
\makeatother
\makeatletter
\@ifpackageloaded{tcolorbox}{}{\usepackage[skins,breakable]{tcolorbox}}
\makeatother
\makeatletter
\@ifundefined{shadecolor}{\definecolor{shadecolor}{rgb}{.97, .97, .97}}
\makeatother
\makeatletter
\makeatother
\makeatletter
\@ifpackageloaded{sidenotes}{}{\usepackage{sidenotes}}
\@ifpackageloaded{marginnote}{}{\usepackage{marginnote}}
\makeatother
\makeatletter
\makeatother
\ifLuaTeX
  \usepackage{selnolig}  % disable illegal ligatures
\fi
\IfFileExists{bookmark.sty}{\usepackage{bookmark}}{\usepackage{hyperref}}
\IfFileExists{xurl.sty}{\usepackage{xurl}}{} % add URL line breaks if available
\urlstyle{same} % disable monospaced font for URLs
\hypersetup{
  pdftitle={The impact of the pilot-UHC: a differences-in-difference approach},
  pdfauthor={Stanley Sayianka},
  colorlinks=true,
  linkcolor={blue},
  filecolor={Maroon},
  citecolor={Blue},
  urlcolor={Blue},
  pdfcreator={LaTeX via pandoc}}

\title{The impact of the pilot-UHC: a differences-in-difference
approach}
\author{Stanley Sayianka}
\date{2024-03-22}

\begin{document}
\maketitle
\ifdefined\Shaded\renewenvironment{Shaded}{\begin{tcolorbox}[breakable, borderline west={3pt}{0pt}{shadecolor}, interior hidden, enhanced, boxrule=0pt, frame hidden, sharp corners]}{\end{tcolorbox}}\fi

The author is grateful to Angela Langat\sidenote{\footnotesize Link to her socials}
for her helpful comments in the analysis and writing of this article.

\hypertarget{introduction}{%
\section{Introduction}\label{introduction}}

To mark Kenya's commitment to the SDG 3 goal on ensuring healthy lives
and promoting well-being for all, the government of Kenya introduced the
Universal Health Coverage (UHC) in December 2018\sidenote{\footnotesize Nzwili,
  Fredrick. 2018. ``Kenyan President Launches Benchmark Universal Health
  Coverage Pilot, To Become Nationwide In 18 Months.'' Health Policy
  Watch (blog). 2018. https://www.healthpolicy-
  watch.org/kenyan-president-launches-
  benchmark-universal-health-coverage-pilot-
  to-become-nationwide-in-18-months/}. The aim of the UHC was to
strengthen primary health care in Kenya by ensuring that citizens have
access to a progressive health benefit and increase the availability and
readiness of key health interventions.

The first phase of the UHC was conducted in a pilot project targeting 4
counties of Kenya: Isiolo, Kisumu, Machakos and Nyeri. The four counties
were chosen on the basis that: they are characterized by high incidence
of both communicable and non-communicable diseases especially diabetes
and hypertension, high population density, high maternal mortality
rates, and high incidence of road traffic injuries. Isiolo was chosen to
experiment with how the UHC would fare in a majorly pastoral county.

The UHC was funded by the national government and a directive was given
to the county governments to abolish fees in the level 4 and 5
facilities. The government would then work out a reimbursement plan
using conditional grants with the county governments for the forgone
fees.

The pilot phase of the UHC was intended to run for a year. In this, the
government intended to gain lessons and insights that would be useful in
scaling up the UHC to the remaining 43 counties nation-wide. Later in
February 2022, the president Uhuru Kenyatta rolled out the UHC across
all counties after a successful pilot.

\begin{center}\rule{0.5\linewidth}{0.5pt}\end{center}

In this analysis, we set out to investigate the impact of the UHC pilot
programme on health accessibility in Kenya. Specifically: what was the
impact of the pilot UHC programme on health acessibility in maternal
health indicators such as: ANC attendance and skilled delivery ?

The data sources available for this study are: the Kenya DHS 2014 and
Kenya DHS 2022. The two are chosen as:

(i). they offer good data on a lot of maternal and child health
indicators

(ii). they take into account the pre and post UHC period, as the 2014
DHS covers the period 2010-2014 (pre UHC) and the 2022 DHS covers the
2020-2022 (post UHC) period. This is useful in studying the impact of
the health intervention (rolling out of the UHC).



\end{document}
